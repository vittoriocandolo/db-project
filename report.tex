\documentclass{article}

\usepackage[utf8]{inputenc}
\usepackage{xcolor}
\usepackage{graphicx}
\usepackage{float}
\usepackage[margin=0.8in]{geometry}
\usepackage[italian]{babel}
\usepackage{hyperref}
\usepackage{minted}

\graphicspath{{./images/}}

\title{Relazione Progetto Basi di Dati \\ A.A. 2021 - 2022}
\author{Antonutti Marco (142426),
        \and Candolo Vittorio Giorgio (141879),
        \and Pipan Martin (151699),
        \and Pozzana Matteo (142250)}
\date{Ottobre 2022}
\begin{document}
\maketitle

\newpage
\renewcommand{\contentsname}{Indice}
\tableofcontents

\newpage

\section{Introduzione}
L'elaborato tratta l'attività di progettazione e di implementazione di una base di dati relazionale per il sistema informativo di un'azienda.
\newline
\newline
Vengono analizzate in particolare le fasi di raccolta e analisi dei requisiti, progettazione concettuale, progettazione logica, progettazione fisica, implementazione e analisi dei dati.

\newpage

\section{Analisi dei requisiti}

\subsection{Requisiti forniti}
Si è acquisita la descrizione del sistema e del dominio dalla consegna.
\newline
\newline
\setlength{\fboxsep}{1em}
\fbox{
    \parbox{\textwidth}{
        Vogliamo modellare le seguenti informazioni riguardanti gli impiegati di un’azienda, i dipartimenti a cui afferiscono, le competenze che possiedono e i progetti a cui partecipano.
        \newline
        \newline
        Ogni impiegato ha una matricola, assegnatagli dalla società, che lo identifica univocamente. Di ogni impiegato interessano il nome e il cognome, la data di nascita e la data di assunzione. Se un impiegato è coniugato con un altro impiegato della stessa società, interessano la data del matrimonio e il coniuge. Ogni impiegato ha una qualifica (ad esempio, segretario, impiegato, programmatore, analista, progettista, ecc.). Dei laureati e dei segretari interessano anche altre informazioni. Dei laureati interessa il tipo di laurea e dei segretari le lingue conosciute.
        \newline
        \newline
        La società è organizzata in dipartimenti. Ciascun dipartimento è identificato univocamente dal nome e possiede un recapito telefonico. Dipartimenti distinti hanno un diverso recapito telefonico. Ogni impiegato afferisce ad un unico dipartimento. Ogni dipartimento viene rifornito da vari fornitori e un fornitore può rifornire vari dipartimenti. Di ogni fornitore interessano il nome e l’indirizzo.
        \newline
        \newline
        I progetti sono identificati da un numero e sono caratterizzati da una città e da un budget. Più impiegati possono essere coinvolti in uno stesso progetto. Un impiegato può partecipare a più progetti, ma può essere assegnato ad un unico progetto per città. Di ogni città con almeno un progetto, interessano il numero di residenti e la regione di appartenenza. Un impiegato può avere più competenze, ma usarne solo alcune per un particolare progetto. Un impiegato usa ogni sua competenza in almeno un progetto. Ad ogni competenza è assegnato un codice, che la identifica univocamente e una descrizione.
    }
}

\subsection{Glossario}
Pur essendo le specifiche iniziali sostanzialmente univoche nell'uso dei termini, si è deciso di redigere comunque un glossario al fine di meglio analizzare le entità e le relazioni fra esse.
\begin{table}[h]
\renewcommand{\arraystretch}{1.5}
\centering
\begin{tabular}{|p{0.11\textwidth}|p{0.45\textwidth}|p{0.1\textwidth}|p{0.25\textwidth}|}
\hline
\textbf{Termine} & \textbf{Descrizione} & \textbf{Sinonimi} & \textbf{Relazioni}        
\\ \hline
Impiegato & Persona che svolge la propria attività professionale presso l'azienda &   & Dipartimento, Progetto, Competenza
\\ \hline
Dipartimento & Struttura organizzativa interna all'azienda che svolge un determinato compito per l'azienda &  & Fornitore, Impiegato
\\ \hline
Fornitore & Azienda esterna, rifornisce i dipartimenti &  & Dipartimento
\\ \hline
Progetto & Serie di attività svolte dagli impiegati dell'azienda al fine di raggiungere un determinato obiettivo aziendale &    & Città, Competenza, Impiegato
\\ \hline
Città & Centro abitato dove l'azienda svolge i suoi progetti &   & Progetto
\\ \hline
Competenza & Abilità di un impiegato, viene usata per svolgere determinati compiti aziendali &   & Progetto, Impiegato 
\\ \hline
Laureato & Impiegato in possesso di una o più lauree &  & Impiegato (\textit{Specializzazione})
\\ \hline
Segretario & Impiegato che svolge compiti amministrativi o di segreteria & & Impiegato (\textit{Specializzazione})
\\ \hline
\end{tabular}
\end{table}

\newpage

\subsection{Strutturazione dei requisiti}
Eventuali integrazioni alla descrizione fornita o assunzioni fatte sono indicate in \textit{corsivo}.
\begin{table}[H]
\renewcommand{\arraystretch}{1.4}
\centering
\begin{tabular}{|p{1\textwidth}|}
\hline
\multicolumn{1}{|c|}{\textbf{Impiegato}}
\\ \hline
\begin{itemize}
\item Identificato univocamente da una matricola e caratterizzato da nome, cognome, data di nascita, data di assunzione e qualifica.
\newline
\item Se coniugato con altro impiegato è ulteriormente caratterizzato dalla data di matrimonio col coniuge.
\newline
\item Se segretario è ulteriormente caratterizzato dalle lingue conosciute.
\newline
\item Se laureato è ulteriormente caratterizzato dal tipo di laurea.
\newline
\item Ogni impiegato afferisce ad un unico dipartimento.
\newline
\item Ogni impiegato può partecipare a più progetti ma ad uno solo per città.
\newline
\item Ogni impiegato usa ogni sua competenza in almeno un progetto.
\end{itemize}
\\ \hline
\multicolumn{1}{|c|}{\textbf{Dipartimento}}
\\ \hline
\begin{itemize}
\item Identificato univocamente dal nome e caratterizzato da un recapito telefonico univoco.
\end{itemize}
\\ \hline
\multicolumn{1}{|c|}{\textbf{Fornitore}}
\\ \hline
\begin{itemize}
\item \textit{Identificato univocamente dalla partita IVA} e caratterizzato dal nome e dall'indirizzo.
\newline
\item Ogni fornitore può rifornire diversi dipartimenti.
\end{itemize}
\\ \hline
\multicolumn{1}{|c|}{\textbf{Progetto}}
\\ \hline
\begin{itemize}
\item Identificato univocamente da un numero e caratterizzato dalla città dove si svolge e da un budget.
\end{itemize}
\\ \hline
\multicolumn{1}{|c|}{\textbf{Città}}
\\ \hline
\begin{itemize}
\item \textit{Identificata univocamente dal nome e dalla regione} e caratterizzata da numero di residenti e regione di appartenenza.
\end{itemize}
\\ \hline
\multicolumn{1}{|c|}{\textbf{Competenza}}
\\ \hline
\begin{itemize}
\item Identificata univocamente da un codice e caratterizzata da una descrizione.
\end{itemize}
\\ \hline
\end{tabular}
\end{table}

\newpage

\subsection{Rappresentazione dei concetti}
Ogni {\color{red} impiegato} ha una {\color{magenta}matricola}, assegnatagli dalla società, che lo identifica univocamente.
\newline
Di ogni impiegato interessano il {\color{magenta}nome} e il {\color{magenta}cognome}, la {\color{magenta}data di nascita} e la {\color{magenta}data di assunzione}.
\newline
Se un impiegato è {\color{blue}coniugato} con un altro impiegato della stessa società, interessano la {\color{teal}data del matrimonio} e il coniuge.
\newline
Ogni impiegato ha una {\color{magenta}qualifica} (ad esempio, segretario, impiegato, programmatore, analista, progettista, ecc.).
\newline
Dei {\color{orange}laureati} e dei {\color{orange}segretari} interessano anche altre informazioni.
\newline
Dei laureati interessa il {\color{magenta}tipo di laurea} e dei segretari le {\color{magenta}lingue conosciute}.
\newline
\newline
La società è organizzata in {\color{red} dipartimenti}.
\newline
Ciascun dipartimento è identificato univocamente dal {\color{magenta}nome} e possiede uno {\color{magenta}recapito telefonico}.
\newline
Dipartimenti distinti hanno un diverso recapito telefonico.
\newline
Ogni impiegato {\color{blue}afferisce} ad un unico dipartimento. 
\newline
Ogni dipartimento viene {\color{blue}rifornito} da vari {\color{red} fornitori} e un fornitore può rifornire vari dipartimenti.
\newline
Di ogni fornitore interessano il {\color{magenta}nome} e l’{\color{magenta}indirizzo}.
\newline
\newline
I {\color{red} progetti} sono identificati da un {\color{magenta}numero} e sono caratterizzati da una {\color{blue}città} e da un {\color{magenta}budget}.
\newline
Più impiegati possono essere coinvolti in uno stesso progetto.
\newline
Un impiegato può {\color{blue}partecipare} a più progetti, ma può essere assegnato ad un unico progetto per {\color{red} città}.
\newline
Di ogni città con almeno un progetto, interessano il {\color{magenta}numero} di residenti e la {\color{magenta}regione} di appartenenza.
\newline
Un impiegato può {\color{blue}avere} più {\color{red} competenze}, ma usarne solo alcune per un particolare progetto.
\newline
Un impiegato usa ogni sua competenza in almeno un progetto.
\newline
Ad ogni competenza è assegnato un {\color{magenta}codice}, che la identifica univocamente, e una {\color{magenta}descrizione}.

\subsubsection*{Legenda}
{\color{red}Entità}
\newline
{\color{orange}Specializzazione}
\newline
{\color{magenta}Attributo}
\newline
{\color{blue}Relazione}
\newline
{\color{teal}Attributo di relazione}

\subsection{Integrazioni e riepilogo delle assunzioni fatte}
In conclusione della fase di analisi dei requisiti si è provveduto ad integrare la consegna di due attributi derivati, al fine di poter svolgere le opportune analisi e modellazioni nelle fasi successive del progetto:
\newline
\newline
Dipartimento - numero fornitori.
\newline
Impiegato - numero progetti.
\newline
\newline
Si vogliono inoltre ricapitolare le assunzioni e le scelte fatte per escludere eventuali ambiguità:
\newline
\newline
Le competenze degli impiegati costituiscono un'entità e sono da intendersi come applicazioni delle qualifiche.
\newline
Le qualifiche invece rimangono attributo degli impiegati come richiesto dalla consegna.
\newline
\newline
Laureati e segretari sono specializzazioni dell'entità Impiegato.
\newline
Tipo di laurea e lingue conosciute sono, rispettivamente, attributi della prima e della seconda specializzazione.
\newline

\newpage

\section{Progettazione Concettuale}

\subsection{Diagramma ER}
\includegraphics[width=\textwidth]{er.png}

\subsubsection*{Legenda}
\includegraphics[width=.22\textwidth]{multivalore.png}
\newline
\includegraphics[width=.22\textwidth]{composto.png}

\newpage

\subsection{Vincoli aziendali}
Al fine di codificare i vincoli imposti dai requisiti del progetto si è definito il seguente vincolo aziendale:
\newline
\newline
Un impiegato può partecipare a più progetti, ma non a più di uno per città.

\subsection{Regole di derivazione}
Si esplicitano poi le regole di derivazione per gli attributi derivati introdotti:
\newline
\newline
\textbf{Dipendenti: numero progetti}
\newline
Conta il numero di tuple della relazione partecipa in cui compare l'impiegato
\newline
\newline
\textbf{Dipartimento: numero fornitori}
\newline
Conta il numero di tuple della relazione rifornisce in cui compare il dipartimento

\newpage

\section{Progettazione Logica}

\subsection{Analisi delle ridondanze}

\subsubsection{Analisi dei cicli}
Lo schema ER non presenta cicli fatta eccezione per la relazione ricorsiva.
\newline
Tuttavia la configurazione scelta prevede il vincolo aziendale descritto in precedenza.

\subsubsection{Requisiti operazionali}
Partendo da quanto descritto nei requisiti e facendo delle assunzioni al fine di poter svolgere poi delle analisi significative, di seguito vengono formulate le principali operazioni, ciascuna con la rispettiva frequenza.
\begin{table}[H]
\renewcommand{\arraystretch}{1.5}
\centering
\begin{tabular}{|p{0.55\textwidth}|l|l|l|}
\cline{1-3}
Operazione & Tipo & Frequenza\\ \cline{1-3}
Ricerca dei fornitori di un dipartimento & Interattiva & 100/mese \\ \cline{1-3}
Ricerca dei segretari che conoscono una determinata lingua & Interattiva & 15/mese\\ \cline{1-3}
Ricerca degli impiegati con determinata competenza & Interattiva & 200/mese \\ \cline{1-3}
Inserimento di un nuovo progetto & Interattiva & 20/mese \\ \cline{1-3}
Ricerca delle competenze di un impiegato & Interattiva & 200/mese \\ \cline{1-3}
Ricerca di impiegati coniugati & Interattiva & 10/mese \\ \cline{1-3}
Ricerca di impiegati che partecipano ad un progetto nullo & Batch & 4/mese \\ \cline{1-3}
Assegnazione di un progetto ad un impiegato & Interattiva & 200/mese \\ \cline{1-3}
Assegnazione fornitore a dipartimento & Interattiva & 20/mese \\ \cline{1-3}
Ricerca tipo di laureato & Interattiva & 50/settimana \\ \cline{1-3}
Ricerca della città in cui l'azienda opera & Interattiva & 25/mese \\ \cline{1-3}
Ricerca del numero di dipendenti per una città & Interattiva & 2/settimana \\ \cline{1-3}
Ricerca del numero di progetti a cui lavora un impiegato & Batch & 50/mese \\ \cline{1-3}
Ricerca del numero di fornitori di un dipartimento & Batch & 30/mese \\ \cline{1-3}
\end{tabular}
\end{table}

\subsubsection{Tavola dei volumi}
Si definisce ora la tavola dei volumi al fine di determinare se mantenere o meno gli attributi derivati.
\newline
Si assume lo stato della base di dati dopo 5 anni di utilizzo.
\begin{table}[H]
\renewcommand{\arraystretch}{1.1}
\centering
\begin{tabular}{|p{0.15\textwidth}|l|l|l|}
\cline{1-3}
Concetto & Tipo & Volume \\ \cline{1-3}
Impiegato & E & 3000 \\ \cline{1-3}
Laureato & E & 1000 \\ \cline{1-3}
Segretario & E & 50 \\ \cline{1-3}
Dipartimento & E & 60 \\ \cline{1-3}
Fornitore & E & 100 \\ \cline{1-3}
Progetto & E & 100 \\ \cline{1-3}
Competenza & E & 30 \\ \cline{1-3}
Città & E & 15 \\ \cline{1-3}
Coniugato & R & 20 \\ \cline{1-3}
Afferisce & R & 3000 \\ \cline{1-3}
Rifornisce & R & 250 \\ \cline{1-3}
Partecipa & R & 9000 \\ \cline{1-3}
Assegnato & R & 100 \\ \cline{1-3}
\end{tabular}
\end{table}

\newpage

\subsubsection{Analisi attributi derivabili: Numero Progetti}
Le operazioni frequenti che coinvolgono questo attributo sono la ricerca del numero di progetti a cui lavora un impiegato e l'assegnazione di un progetto ad un impiegato.
\newline
\newline
\textbf{Lettura senza attributo derivato}
\begin{table}[H]
\renewcommand{\arraystretch}{1.2}
\centering
\begin{tabular}{|p{0.20\textwidth}|l|l|l|l|}
\cline{1-4}
Concetto & Tipo & Accessi & Tipo accesso\\ \cline{1-4}
Partecipa & R & 9000 & R \\ \cline{1-4}
\end{tabular}
\end{table}
\noindent
\textbf{Lettura con attributo derivato}
\begin{table}[H]
\renewcommand{\arraystretch}{1.2}
\centering
\begin{tabular}{|p{0.20\textwidth}|l|l|l|l|}
\cline{1-4}
Concetto & Tipo & Accessi & Tipo accesso\\ \cline{1-4}
Impiegato & E & 1 & R \\ \cline{1-4}
\end{tabular}
\end{table}
\noindent
\textbf{Scrittura senza attributo derivato}
\begin{table}[H]
\renewcommand{\arraystretch}{1.2}
\centering
\begin{tabular}{|p{0.20\textwidth}|l|l|l|l|}
\cline{1-4}
Concetto & Tipo & Accessi & Tipo accesso\\ \cline{1-4}
Partecipa & R & 3 & W \\ \cline{1-4}
\end{tabular}
\end{table}
\noindent
\textbf{Scrittura con attributo derivato}
\begin{table}[H]
\renewcommand{\arraystretch}{1.2}
\centering
\begin{tabular}{|p{0.20\textwidth}|l|l|l|l|}
\cline{1-4}
Concetto & Tipo & Accessi & Tipo accesso\\ \cline{1-4}
Partecipa & R & 3 & W \\ \cline{1-4}
Impiegato & E & 1 & W \\ \cline{1-4}
\end{tabular}
\end{table}
Applicando alle scritture un peso doppio rispetto alle letture si ottengono i valori di seguito riportati.
\begin{table}[H]
\renewcommand{\arraystretch}{1.2}
\centering
\begin{tabular}{|p{0.20\textwidth}|l|l|l|}
\cline{1-3}
& Costo \textbf{senza} attributo & Costo \textbf{con} attributo \\ \cline{1-3}
Costo lettura & 450000 & 50 \\ \cline{1-3}
Costo scrittura & 1200 & 1600 \\ \cline{1-3}
Costo totale & 45120 & 1650 \\ \cline{1-3}
\end{tabular}
\end{table}
Si conclude quindi di mantenere l'attributo derivato vista la netta prevalenza delle operazioni di lettura.

\newpage

\subsubsection{Analisi attributi derivabili: Numero Fornitori}
Le operazioni frequenti che coinvolgono questo attributo sono la ricerca del numero di fornitori di un dipartimento e l'assegnazione di un fornitore ad un dipartimento.
\newline
\newline
\textbf{Lettura senza attributo derivato}
\begin{table}[H]
\renewcommand{\arraystretch}{1.2}
\centering
\begin{tabular}{|p{0.20\textwidth}|l|l|l|l|}
\cline{1-4}
Concetto & Tipo & Accessi & Tipo accesso\\ \cline{1-4}
Rifornisce & R & 250 & R \\ \cline{1-4}
\end{tabular}
\end{table}
\noindent
\textbf{Lettura con attributo derivato}
\begin{table}[H]
\renewcommand{\arraystretch}{1.2}
\centering
\begin{tabular}{|p{0.20\textwidth}|l|l|l|l|}
\cline{1-4}
Concetto & Tipo & Accessi & Tipo accesso\\ \cline{1-4}
Rifornisce & R & 1 & W \\ \cline{1-4}
Dipartimento & E & 1 & W \\ \cline{1-4}
\end{tabular}
\end{table}
\noindent
\textbf{Scrittura senza attributo derivato}
\begin{table}[H]
\renewcommand{\arraystretch}{1.2}
\centering
\begin{tabular}{|p{0.20\textwidth}|l|l|l|l|}
\cline{1-4}
Concetto & Tipo & Accessi & Tipo accesso\\ \cline{1-4}
Rifornisce & R & 1 & W \\ \cline{1-4}
\end{tabular}
\end{table}
\noindent
\textbf{Scrittura con attributo derivato}
\begin{table}[H]
\renewcommand{\arraystretch}{1.2}
\centering
\begin{tabular}{|p{0.20\textwidth}|l|l|l|l|}
\cline{1-4}
Concetto & Tipo & Accessi & Tipo accesso\\ \cline{1-4}
Dipartimento & E & 1 & R \\ \cline{1-4}
\end{tabular}
\end{table}
Applicando alle scritture un peso doppio rispetto alle letture si ottengono i valori di seguito riportati.
\begin{table}[H]
\renewcommand{\arraystretch}{1.2}
\centering
\begin{tabular}{|p{0.20\textwidth}|l|l|l|}
\cline{1-3}
& Costo \textbf{senza} attributo & Costo \textbf{con} attributo \\ \cline{1-3}
Costo lettura & 7500 & 30 \\ \cline{1-3}
Costo scrittura & 40 & 80 \\ \cline{1-3}
Costo totale & 7540 & 110 \\ \cline{1-3}
\end{tabular}
\end{table}
Si conclude quindi di mantenere l'attributo derivato vista la netta prevalenza delle operazioni di lettura.
\newpage

\subsection{Ristrutturazione del diagramma ER}

\subsubsection{Eliminazione delle generalizzazioni}
Il diagramma ER presenta due specializzazioni per l'entità impiegato:
\newline
\newline
\includegraphics[width=0.5\textwidth]{er_R1.png}
\newline
\newline
Gli attributi caratteristici delle due classi di specializzazione vengono assegnati all'entità padre.
\newline
Si rinomina l'attributo laurea in tipo laurea.
\newline
\newline
\includegraphics[width=0.5\textwidth]{er_R2.png}

\subsubsection{Eliminazione degli attributi multivalore}
É ora necessario eliminare gli attributi multivalore.
\newline
Gli unici presenti sono gli attributi "lingue" e "tipo laurea" di impiegato.
\newline
Si procede ad una reificazione.
\newline
\newline
Nel caso di tipo laurea si è deciso inoltre di separare la materia dal livello (tipo).
\newline
\newline
\includegraphics[width=0.5\textwidth]{er_R3.png}

\subsubsection{Eliminazione degli attributi composti}
L'unico attributo composto che compare nel diagramma ER è l'indirizzo dell'entità fornitore.
\newline
Si è scelto di scorporare il summenzionato attributo nei campi componenti.

\newpage

\subsection{Diagramma ER Ristrutturato}
\includegraphics[width=\textwidth]{er_R.png}

\newpage

\subsection{Traduzione al modello relazionale}

\subsubsection{Traduzione delle entità}
\textbf{Fornitore} (\underline{PartitaIva}, Nome, Via, Civico, Città)
\newline
\newline
\textbf{Dipartimento} (\underline{Nome}, RecapitoTelefonico, NumeroFornitori)
\newline
- RecapitoTelefonico attributo dal valore unico
\newline
- NumeroFornitori attributo derivato
\newline
\newline
\textbf{Impiegato} (\underline{Matricola}, Nome, Cognome, DataDiNascita, DataDiAssunzione, Dipartimento, Qualifica, NumeroProgetti)
\newline
- Dipartimento chiave esterna (riferimento alla chiave primaria dell'entità Dipartimento)
\newline
- NumeroProgetti attributo derivato
\newline
\newline
\textbf{Competenza} (\underline{Codice}, Descrizione)
\newline
- Descrizione unico
\newline
\newline
\textbf{Città} (\underline{Nome, Regione}, NumeroDiResidenti)
\newline
\newline
\textbf{Progetto} (\underline{Numero}, Budget, Città, Regione)
\newline
- Città chiave esterna (riferimento alla chiave primaria dell'entità Città)
\newline
- Regione chiave esterna (riferimento alla chiave primaria dell'entità Città)
\newline
\newline
\textbf{Tipo Laurea}
\newline
Si è deciso di codificare l'informazione direttamente come attributi della relazione Laureato.
\newline
\newline
\textbf{Lingua}
\newline
Si è deciso di codificare l'informazione direttamente come attributo della relazione Segretario.

\subsubsection{Traduzione delle relazioni}
\textbf{Rifornisce} (\underline{Dipartimento, Fornitore})
\newline
- Dipartimento chiave esterna (riferimento alla chiave primaria dell'entità Dipartimento)
\newline
- Fornitore chiave esterna (riferimento alla chiave primaria dell'entità Fornitore)
\newline
\newline
\textbf{Segretario} (\underline{Impiegato, Lingua}) [Rinomina di Conosce Lingua]
\newline
- Impiegato (riferimento alla chiave primaria dell'entità Impiegato)
\newline
- Lingua chiave esterna (riferimento alla chiave primaria dell'entità Lingua)
\newline
\newline
\textbf{Matrimonio} (\underline{Marito}, Moglie, DataDiMatrimonio) [Rinomina di Coniuge]
\newline
- Marito chieve esterna (riferimento alla chiave primaria dell'entità Impiegato)
\newline
- Moglie chieve esterna (riferimento alla chiave primaria dell'entità Impiegato)
\newline
- Moglie attributo dal valore unico
\newline
- Marito diverso da Moglie
\newline
\newline
\textbf{Laureato} (\underline{Impiegato, TipoLaurea, Materia}) [Rinomina di Possiede Laurea]
\newline
- Impiegato chiave esterna (riferimento alla chiave primaria dell'entità Impiegato)
\newline
\newline
\textbf{Partecipa} (\underline{Impiegato, Competenza, Progetto})
\newline
- Impiegato chiave esterna (riferimento alla chiave primaria dell'entità Impiegato)
\newline
- Competenza chiave esterna (riferimento alla chiave primaria dell'entità Competenza)
\newline
- Progetto chiave esterna (riferimento alla chiave primaria dell'entità Progetto)
\newline
\newline
\textbf{Afferisce}
\newline
Si è deciso di codificare l'informazione direttamente come attributo dell'entità Impiegato.

\newpage

\subsection{Diagramma relazionale}
\includegraphics[width=1\textwidth]{r.png}

\subsection{Osservazioni}

\subsubsection{Vincolo sulle partecipazioni}
Il \hyperlink{page.9}{vincolo aziendale} relativo a progetti, impiegati e città non è codificabile nello schema.

\subsubsection{Vincolo sugli impiegati segretari}
Dato che a differenza di quanto accade per gli impiegati laureati, i requisiti forniti richiedono che segretario sia uno dei possibili valori dell'attributo \textit{Qualifica} di \textbf{Impiegato}, è necessario implementare un ulteriore vincolo che garantisca che gli impiegati con \textit{Qualifica} "Segretario" compaiano nella tabella \textbf{Segretario}.
\newline
\newline
Sono indicati in corsivo i nomi di attributo e in grassetto i nomi delle tabelle.

\newpage

\section{Progettazione Fisica}

\subsection{Indici}
Si valuta in questa fase l'introduzione di indici per ottimizzare le prestazioni della base di dati.
\newline
\newline
Noto che le chiavi primarie e gli attributi unici delle tabelle sono già indicizzati dal DBMS PostreSQL si è ritenuto di studiare l'uso di indici sugli attributi \textit{Competenza} e \textit{DataDiAssunzione} della tabella \textbf{Impiegato} e sull'attributo \textit{Budget} della tabella \textbf{Progetto}.
\newline
\newline
Per valutare l'impatto degli indici sulle operazioni di lettura e scrittura si è fatto uso come tool di profiling delle interrogazioni del comando EXPLAIN ANALISE.
\newline
\newline
Tramite una semplice funzione vengono raccolti 50 tempi, con e senza indice.
\newline
Per motivi di presentazione nelle tabelle ne vengono mostrati 30 mentre tutti e 50 contribuiscono ai grafici.
\newline
A margine saranno presenti anche alcune valutazioni sulla dimensione delle tabelle.

\subsubsection{Valutazione dell'indicizzazione di qualifica di impiegato}
\textbf{Operazioni di selezione}
\begin{minted}{sql}
Interrogazione: explain analyse select * from impiegato where qualifica='Programmatore'
\end{minted}
\begin{table}[H]
\renewcommand{\arraystretch}{1.1}
\centering
\begin{tabular}{|p{4cm}|p{4cm}|p{0cm}|p{4cm}|p{4cm}|}
\cline{1-5}
Planning \textbf{senza} indice & Execution \textbf{senza} indice & & Planning \textbf{con} indice & Execution \textbf{con} indice \\ \cline{1-5}
0.048 & 0.449 & & 0.057 & 0.373 \\ \cline{1-5}
0.022 & 0.366 & & 0.029 & 0.342 \\ \cline{1-5}
0.017 & 0.299 & & 0.03 & 0.328 \\ \cline{1-5}
0.017 & 0.329 & & 0.021 & 0.285 \\ \cline{1-5}
0.017 & 0.324 & & 0.023 & 0.293 \\ \cline{1-5}
0.019 & 0.287 & & 0.028 & 0.31 \\ \cline{1-5}
0.02 & 0.318 & & 0.028 & 0.282 \\ \cline{1-5}
0.035 & 0.367 & & 0.02 & 0.285 \\ \cline{1-5}
0.019 & 0.299 & & 0.022 & 0.337 \\ \cline{1-5}
0.023 & 0.344 & & 0.03 & 0.287 \\ \cline{1-5}
0.018 & 0.304 & & 0.028 & 0.291 \\ \cline{1-5}
0.045 & 0.494 & & 0.029 & 0.288 \\ \cline{1-5}
0.023 & 0.344 & & 0.059 & 0.431 \\ \cline{1-5}
0.02 & 0.333 & & 0.033 & 0.312 \\ \cline{1-5}
0.02 & 0.313 & & 0.021 & 0.288 \\ \cline{1-5}
0.026 & 0.302 & & 0.02 & 0.292 \\ \cline{1-5}
0.025 & 0.317 & & 0.029 & 0.307 \\ \cline{1-5}
0.017 & 0.285 & & 0.028 & 0.28 \\ \cline{1-5}
0.017 & 0.283 & & 0.02 & 0.288 \\ \cline{1-5}
0.017 & 0.279 & & 0.023 & 0.31 \\ \cline{1-5}
0.016 & 0.278 & & 0.031 & 0.294 \\ \cline{1-5}
0.017 & 0.277 & & 0.02 & 0.29 \\ \cline{1-5}
0.017 & 0.311 & & 0.026 & 0.29 \\ \cline{1-5}
0.017 & 0.284 & & 0.029 & 0.285 \\ \cline{1-5}
0.016 & 0.277 & & 0.02 & 0.291 \\ \cline{1-5}
0.017 & 0.306 & & 0.02 & 0.288 \\ \cline{1-5}
0.023 & 0.27 & & 0.021 & 0.306 \\ \cline{1-5}
0.047 & 0.433 & & 0.02 & 0.293 \\ \cline{1-5}
0.029 & 0.365 & & 0.021 & 0.289 \\ \cline{1-5}
0.021 & 0.367 & & 0.027 & 0.303 \\ \cline{1-5}
\end{tabular}
\end{table}

\newpage
\noindent
\textbf{Operazioni di modifica}
\begin{minted}{sql}
Interrogazione:
explain analyse update impiegato set qualifica='Analista' where numero_progetti between 2 and 3
\end{minted}
\begin{table}[H]
\renewcommand{\arraystretch}{1.1}
\centering
\begin{tabular}{|p{4cm}|p{4cm}|p{0cm}|p{4cm}|p{4cm}|}
\cline{1-5}
Planning \textbf{senza} indice & Execution \textbf{senza} indice & & Planning \textbf{con} indice & Execution \textbf{con} indice \\ \cline{1-5}
0.07 & 5.044 & & 0.072 & 4.841 \\ \cline{1-5}
0.069 & 6.875 & & 0.074 & 4.988 \\ \cline{1-5}
0.064 & 5.414 & & 0.062 & 4.992 \\ \cline{1-5}
0.094 & 6.242 & & 0.065 & 4.801 \\ \cline{1-5}
0.067 & 4.806 & & 0.111 & 5.125 \\ \cline{1-5}
0.086 & 4.958 & & 0.055 & 5.225 \\ \cline{1-5}
0.058 & 4.586 & & 0.046 & 4.821 \\ \cline{1-5}
0.04 & 4.538 & & 0.143 & 5.447 \\ \cline{1-5}
0.077 & 5.114 & & 0.086 & 5.777 \\ \cline{1-5}
0.049 & 4.981 & & 0.077 & 5.946 \\ \cline{1-5}
0.065 & 4.731 & & 0.058 & 5.284 \\ \cline{1-5}
0.048 & 4.92 & & 0.051 & 5.815 \\ \cline{1-5}
0.065 & 4.819 & & 0.152 & 5.433 \\ \cline{1-5}
0.042 & 4.507 & & 0.071 & 5.827 \\ \cline{1-5}
0.043 & 4.755 & & 0.05 & 5.552 \\ \cline{1-5}
0.035 & 4.589 & & 0.041 & 5.236 \\ \cline{1-5}
0.039 & 4.669 & & 0.05 & 5.247 \\ \cline{1-5}
0.043 & 4.574 & & 0.049 & 5.308 \\ \cline{1-5}
0.032 & 4.493 & & 0.05 & 5.039 \\ \cline{1-5}
0.039 & 4.474 & & 0.045 & 5.247 \\ \cline{1-5}
0.04 & 4.771 & & 0.035 & 5.23 \\ \cline{1-5}
0.04 & 4.49 & & 0.059 & 5.107 \\ \cline{1-5}
0.031 & 4.547 & & 0.036 & 4.863 \\ \cline{1-5}
0.031 & 4.625 & & 0.033 & 5.094 \\ \cline{1-5}
0.031 & 4.571 & & 0.035 & 4.873 \\ \cline{1-5}
0.039 & 4.441 & & 0.033 & 4.976 \\ \cline{1-5}
0.04 & 4.622 & & 0.033 & 4.998 \\ \cline{1-5}
0.041 & 4.485 & & 0.033 & 4.88 \\ \cline{1-5}
0.032 & 4.657 & & 0.033 & 4.853 \\ \cline{1-5}
0.043 & 4.61 & & 0.044 & 5.175 \\ \cline{1-5}
\end{tabular}
\end{table}
\noindent
\newline
\textbf{Considerazioni sullo spazio occupato}
\newline
\newline
Sfruttando le seguenti interrogazioni è possibile conoscere lo spazio occupato dalla tabella e dagli indici:
\begin{minted}{sql}
SELECT pg_size_pretty(pg_total_relation_size ('impiegato'));
SELECT pg_size_pretty(pg_indexes_size ('impiegato'));
\end{minted}
Per impiegato il risultato è di \textbf{2128kB} prima dell'indicizzazione e \textbf{2224kB} a seguito di essa con una dimensione dell'indice di \textbf{96kB}.

\newpage
\noindent
\newline
Concludiamo visualizzando i risultati della profilazione tramite box plot realizzati in R:
\newline
\newline
\textbf{Operazioni di selezione}
\newline
\newline
\includegraphics[width=0.5\textwidth]{planning_impiegato_qualifica_selezione.png}
\includegraphics[width=0.5\textwidth]{execution_impiegato_qualifica_selezione.png}
\newline
\newline
\textbf{Operazioni di modifica}
\newline
\newline
\includegraphics[width=0.5\textwidth]{planning_impiegato_qualifica_modifica.png}
\includegraphics[width=0.5\textwidth]{execution_impiegato_qualifica_modifica.png}
\newline
\newline
Visto l'esito del profiling si ritiene di implementare l'indice.

\newpage

\subsubsection{Valutazione dell'indicizzazione di data di assunzione di impiegato}
\textbf{Operazioni di selezione}
\begin{minted}{sql}
Interrogazione: explain analyse select * from impiegato where data_di_assunzione<='20000101'
\end{minted}
\begin{table}[H]
\renewcommand{\arraystretch}{1.1}
\centering
\begin{tabular}{|p{4cm}|p{4cm}|p{0cm}|p{4cm}|p{4cm}|}
\cline{1-5}
Planning \textbf{senza} indice & Execution \textbf{senza} indice & & Planning \textbf{con} indice & Execution \textbf{con} indice \\ \cline{1-5}
0.084 & 0.896 & & 0.092 & 0.014 \\ \cline{1-5}
0.04 & 0.778 & & 0.053 & 0.019 \\ \cline{1-5}
0.026 & 0.356 & & 0.06 & 0.01 \\ \cline{1-5}
0.036 & 0.377 & & 0.061 & 0.01 \\ \cline{1-5}
0.024 & 0.33 & & 0.061 & 0.01 \\ \cline{1-5}
0.036 & 0.338 & & 0.059 & 0.01 \\ \cline{1-5}
0.052 & 0.452 & & 0.064 & 0.012 \\ \cline{1-5}
0.026 & 0.548 & & 0.068 & 0.011 \\ \cline{1-5}
0.116 & 1.23 & & 0.117 & 0.014 \\ \cline{1-5}
0.054 & 0.471 & & 0.08 & 0.012 \\ \cline{1-5}
0.03 & 0.324 & & 0.061 & 0.01 \\ \cline{1-5}
0.028 & 0.293 & & 0.081 & 0.027 \\ \cline{1-5}
0.018 & 0.277 & & 0.061 & 0.011 \\ \cline{1-5}
0.019 & 0.294 & & 0.09 & 0.029 \\ \cline{1-5}
0.018 & 0.271 & & 0.072 & 0.013 \\ \cline{1-5}
0.029 & 0.386 & & 0.06 & 0.01 \\ \cline{1-5}
0.019 & 0.27 & & 0.059 & 0.011 \\ \cline{1-5}
0.036 & 0.465 & & 0.112 & 0.021 \\ \cline{1-5}
0.02 & 0.331 & & 0.089 & 0.022 \\ \cline{1-5}
0.03 & 0.316 & & 0.065 & 0.011 \\ \cline{1-5}
0.04 & 0.524 & & 0.064 & 0.014 \\ \cline{1-5}
0.027 & 0.366 & & 0.103 & 0.013 \\ \cline{1-5}
0.033 & 0.539 & & 0.107 & 0.015 \\ \cline{1-5}
0.029 & 0.341 & & 0.095 & 0.023 \\ \cline{1-5}
0.027 & 0.352 & & 0.046 & 0.01 \\ \cline{1-5}
0.037 & 0.317 & & 0.051 & 0.009 \\ \cline{1-5}
0.034 & 0.384 & & 0.076 & 0.01 \\ \cline{1-5}
0.052 & 0.45 & & 0.075 & 0.016 \\ \cline{1-5}
0.027 & 0.428 & & 0.071 & 0.012 \\ \cline{1-5}
0.045 & 0.711 & & 0.08 & 0.012 \\ \cline{1-5}
\end{tabular}
\end{table}

\newpage
\noindent
\textbf{Operazioni di modifica}
\begin{minted}{sql}
Interrogazione:
explain analyse 
    update impiegato 
        set data_di_assunzione='20220829' where data_di_nascita between '19820101' and '20050101'
\end{minted}
\begin{table}[H]
\renewcommand{\arraystretch}{1.2}
\centering
\begin{tabular}{|p{4cm}|p{4cm}|p{0cm}|p{4cm}|p{4cm}|}
\cline{1-5}
Planning \textbf{senza} indice & Execution \textbf{senza} indice & & Planning \textbf{con} indice & Execution \textbf{con} indice \\ \cline{1-5}
0.069 & 7.221 & & 0.073 & 7.519 \\ \cline{1-5}
0.143 & 7.546 & & 0.066 & 6.17 \\ \cline{1-5}
0.097 & 7.623 & & 0.055 & 6.281 \\ \cline{1-5}
0.124 & 7.475 & & 0.107 & 7.559 \\ \cline{1-5}
0.072 & 7.268 & & 0.068 & 7.164 \\ \cline{1-5}
0.068 & 7.978 & & 0.072 & 6.133 \\ \cline{1-5}
0.113 & 12.945 & & 0.101 & 6.604 \\ \cline{1-5}
0.065 & 6.818 & & 0.051 & 6.122 \\ \cline{1-5}
0.046 & 6.371 & & 0.046 & 6.782 \\ \cline{1-5}
0.043 & 6.328 & & 0.048 & 6.263 \\ \cline{1-5}
0.044 & 6.414 & & 0.035 & 6.486 \\ \cline{1-5}
0.074 & 8.316 & & 0.037 & 6.531 \\ \cline{1-5}
0.081 & 6.918 & & 0.037 & 6.479 \\ \cline{1-5}
0.062 & 7.582 & & 0.047 & 6.468 \\ \cline{1-5}
0.081 & 6.874 & & 0.036 & 6.444 \\ \cline{1-5}
0.046 & 6.597 & & 0.034 & 6.547 \\ \cline{1-5}
0.037 & 6.201 & & 0.049 & 6.548 \\ \cline{1-5}
0.046 & 6.235 & & 0.043 & 6.966 \\ \cline{1-5}
0.083 & 4.914 & & 0.071 & 6.882 \\ \cline{1-5}
0.029 & 4.792 & & 0.084 & 6.668 \\ \cline{1-5}
0.028 & 4.691 & & 0.083 & 6.987 \\ \cline{1-5}
0.027 & 4.609 & & 0.098 & 6.878 \\ \cline{1-5}
0.034 & 4.585 & & 0.08 & 6.891 \\ \cline{1-5}
0.035 & 4.651 & & 0.079 & 6.878 \\ \cline{1-5}
0.026 & 4.521 & & 0.072 & 7.138 \\ \cline{1-5}
0.035 & 4.61 & & 0.091 & 7.488 \\ \cline{1-5}
0.035 & 4.653 & & 0.069 & 7.252 \\ \cline{1-5}
0.045 & 5.002 & & 0.072 & 6.607 \\ \cline{1-5}
0.029 & 4.568 & & 0.073 & 6.82 \\ \cline{1-5}
0.052 & 5.783 & & 0.118 & 7.071 \\ \cline{1-5}
\end{tabular}
\end{table}
\noindent
\newline
\textbf{Considerazioni sullo spazio occupato}
\newline
\newline
Sfruttando le seguenti interrogazioni è possibile conoscere lo spazio occupato dalla tabella e dagli indici:
\begin{minted}{sql}
SELECT pg_size_pretty(pg_total_relation_size ('impiegato'));
SELECT pg_size_pretty(pg_indexes_size ('impiegato'));
\end{minted}
Per impiegato il risultato è di \textbf{2224kB} prima dell'indicizzazione e \textbf{2312kB} a seguito di essa con una dimensione dell'indice di \textbf{88kB}.

\newpage
\noindent
\newline
Concludiamo visualizzando i risultati della profilazione tramite box plot realizzati in R:
\newline
\newline
\textbf{Operazioni di selezione}
\newline
\newline
\includegraphics[width=0.5\textwidth]{planning_impiegato_dataAssunzione_selezione.png}
\includegraphics[width=0.5\textwidth]{execution_impiegato_dataAssunzione_selezione.png}
\newline
\newline
\textbf{Operazioni di modifica}
\newline
\newline
\includegraphics[width=0.5\textwidth]{planning_impiegato_dataAssunzione_modifica.png}
\includegraphics[width=0.5\textwidth]{execution_impiegato_dataAssunzione_modifica.png}
\newline
\newline
Visto l'esito del profiling e data la natura dell'attributo su cui si è costruito l'indice si ritiene di implementarlo.

\newpage

\subsubsection{Valutazione dell'indicizzazione di budget di progetto}
\textbf{Operazioni di selezione}
\begin{minted}{sql}
Interrogazione: explain analyse select * from progetto where budget>=100000
\end{minted}
\begin{table}[H]
\renewcommand{\arraystretch}{1.1}
\centering
\begin{tabular}{|p{4cm}|p{4cm}|p{0cm}|p{4cm}|p{4cm}|}
\cline{1-5}
Planning \textbf{senza} indice & Execution \textbf{senza} indice & & Planning \textbf{con} indice & Execution \textbf{con} indice \\ \cline{1-5}
0.048 & 0.042 & & 0.078 & 0.036 \\ \cline{1-5}
0.041 & 0.048 & & 0.037 & 0.03 \\ \cline{1-5}
0.024 & 0.041 & & 0.05 & 0.022 \\ \cline{1-5}
0.027 & 0.035 & & 0.041 & 0.022 \\ \cline{1-5}
0.019 & 0.033 & & 0.044 & 0.024 \\ \cline{1-5}
0.02 & 0.033 & & 0.059 & 0.046 \\ \cline{1-5}
0.019 & 0.023 & & 0.069 & 0.041 \\ \cline{1-5}
0.02 & 0.03 & & 0.041 & 0.022 \\ \cline{1-5}
0.019 & 0.031 & & 0.041 & 0.022 \\ \cline{1-5}
0.022 & 0.031 & & 0.032 & 0.029 \\ \cline{1-5}
0.037 & 0.041 & & 0.04 & 0.022 \\ \cline{1-5}
0.026 & 0.041 & & 0.046 & 0.026 \\ \cline{1-5}
0.027 & 0.023 & & 0.038 & 0.027 \\ \cline{1-5}
0.027 & 0.042 & & 0.031 & 0.031 \\ \cline{1-5}
0.072 & 0.052 & & 0.035 & 0.018 \\ \cline{1-5}
0.034 & 0.049 & & 0.089 & 0.051 \\ \cline{1-5}
0.04 & 0.042 & & 0.035 & 0.03 \\ \cline{1-5}
0.02 & 0.023 & & 0.045 & 0.033 \\ \cline{1-5}
0.027 & 0.023 & & 0.049 & 0.021 \\ \cline{1-5}
0.027 & 0.047 & & 0.043 & 0.029 \\ \cline{1-5}
0.034 & 0.04 & & 0.044 & 0.033 \\ \cline{1-5}
0.021 & 0.032 & & 0.027 & 0.02 \\ \cline{1-5}
0.02 & 0.031 & & 0.03 & 0.025 \\ \cline{1-5}
0.03 & 0.026 & & 0.034 & 0.018 \\ \cline{1-5}
0.02 & 0.025 & & 0.049 & 0.031 \\ \cline{1-5}
0.032 & 0.028 & & 0.056 & 0.031 \\ \cline{1-5}
0.027 & 0.032 & & 0.044 & 0.026 \\ \cline{1-5}
0.022 & 0.025 & & 0.041 & 0.023 \\ \cline{1-5}
0.027 & 0.037 & & 0.047 & 0.029 \\ \cline{1-5}
0.036 & 0.042 & & 0.061 & 0.045 \\ \cline{1-5}
\end{tabular}
\end{table}

\newpage
\noindent
\textbf{Operazioni di modifica}
\begin{minted}{sql}
Interrogazione: explain analyse update progetto set budget=budget*0.75 where citta='Atlanta'
\end{minted}
\begin{table}[H]
\renewcommand{\arraystretch}{1.2}
\centering
\begin{tabular}{|p{4cm}|p{4cm}|p{0cm}|p{4cm}|p{4cm}|}
\cline{1-5}
Planning \textbf{senza} indice & Execution \textbf{senza} indice & & Planning \textbf{con} indice & Execution \textbf{con} indice \\ \cline{1-5}
0.065 & 0.101 & & 0.055 & 0.088 \\ \cline{1-5}
0.063 & 0.08 & & 0.035 & 0.07 \\ \cline{1-5}
0.046 & 0.079 & & 0.029 & 0.062 \\ \cline{1-5}
0.034 & 0.072 & & 0.027 & 0.061 \\ \cline{1-5}
0.04 & 0.07 & & 0.035 & 0.062 \\ \cline{1-5}
0.036 & 0.064 & & 0.034 & 0.06 \\ \cline{1-5}
0.038 & 0.064 & & 0.027 & 0.06 \\ \cline{1-5}
0.041 & 0.071 & & 0.035 & 0.063 \\ \cline{1-5}
0.045 & 0.085 & & 0.033 & 0.061 \\ \cline{1-5}
0.039 & 0.075 & & 0.033 & 0.061 \\ \cline{1-5}
0.043 & 0.091 & & 0.034 & 0.061 \\ \cline{1-5}
0.034 & 0.084 & & 0.033 & 0.06 \\ \cline{1-5}
0.03 & 0.076 & & 0.035 & 0.062 \\ \cline{1-5}
0.035 & 0.096 & & 0.051 & 0.074 \\ \cline{1-5}
0.026 & 0.057 & & 0.051 & 0.125 \\ \cline{1-5}
0.039 & 0.054 & & 0.046 & 0.077 \\ \cline{1-5}
0.025 & 0.058 & & 0.057 & 0.112 \\ \cline{1-5}
0.024 & 0.054 & & 0.037 & 0.075 \\ \cline{1-5}
0.03 & 0.049 & & 0.037 & 0.067 \\ \cline{1-5}
0.03 & 0.053 & & 0.039 & 0.072 \\ \cline{1-5}
0.024 & 0.054 & & 0.042 & 0.073 \\ \cline{1-5}
0.026 & 0.056 & & 0.103 & 0.072 \\ \cline{1-5}
0.056 & 0.093 & & 0.056 & 0.095 \\ \cline{1-5}
0.038 & 0.074 & & 0.078 & 0.136 \\ \cline{1-5}
0.04 & 0.066 & & 0.052 & 0.125 \\ \cline{1-5}
0.056 & 0.093 & & 0.035 & 0.062 \\ \cline{1-5}
0.042 & 0.075 & & 0.035 & 0.061 \\ \cline{1-5}
0.047 & 0.091 & & 0.026 & 0.064 \\ \cline{1-5}
0.073 & 0.13 & & 0.035 & 0.064 \\ \cline{1-5}
0.05 & 0.112 & & 0.034 & 0.101 \\ \cline{1-5}
\end{tabular}
\end{table}
\noindent
\newline
\textbf{Considerazioni sullo spazio occupato}
\newline
\newline
Sfruttando le seguenti interrogazioni è possibile conoscere lo spazio occupato dalla tabella e dagli indici:
\begin{minted}{sql}
SELECT pg_size_pretty(pg_total_relation_size ('progetto'));
SELECT pg_size_pretty(pg_indexes_size ('progetto'));
\end{minted}
Per impiegato il risultato è di \textbf{72kB} prima dell'indicizzazione e \textbf{88kB} a seguito di essa con una dimensione dell'indice di \textbf{16kB}.

\newpage
\noindent
\newline
Concludiamo visualizzando i risultati della profilazione tramite box plot
\newline
\newline
\textbf{Operazioni di selezione}
\newline
\newline
\includegraphics[width=0.5\textwidth]{planning_progetto_budget_selezione.png}
\includegraphics[width=0.5\textwidth]{execution_progetto_budget_selezione.png}
\newline
\newline
\textbf{Operazioni di modifica}
\newline
\includegraphics[width=0.5\textwidth]{planning_progetto_budget_modifica.png}
\includegraphics[width=0.5\textwidth]{execution_progetto_budget_modifica.png}
\newline
\newline
Visto l'esito del profiling si ritiene di implementare l'indice.

\newpage

\section{Implementazione}

\subsection{Definizione delle enumerazioni}
Si è valutato di sfruttare un'enumerazione per la codifica delle lingue conosciuta dal segretario.
\newline
Questa soluzione garantisce che il valore dell'attributo appartenga ad un insieme prestabilito.
\newline
\newline
I comandi per gestire il tipo enumerazione sono semplici ma è importante notare che PostgreSQL permette l'aggiunta di nuovi valori all'enumerazione e la modifica di quelli esistenti ma non permette di rimuovere valori.
\begin{minted}{sql}
CREATE TYPE lingua AS enum (
  'Inglese', 'Tedesco', 'Francese', 
  'Polacco', 'Russo', 'Arabo', 'Cinese', 
  'Giapponese', 'Coreano', 'Portoghese', 
  'Svedese', 'Indiano', 'Italiano'
);
\end{minted}

\subsection{Definizione delle tabelle}
Seguono le tabelle definite sulla base dello \hyperlink{page.15}{schema relazionale}.
\begin{minted}{sql}
CREATE TABLE fornitore (
  partitaiva numeric(11, 0) PRIMARY KEY, 
  nome varchar(50), 
  via varchar(50), 
  civico integer, 
  citta varchar(30)
);

CREATE TABLE dipartimento (
  nome varchar(50) PRIMARY KEY, 
  recapito_telefonico numeric(10, 0) UNIQUE, 
  numero_fornitori integer NOT NULL DEFAULT 0
);

CREATE TABLE rifornisce (
  dipartimento varchar(50) REFERENCES dipartimento(nome) ON UPDATE CASCADE ON DELETE CASCADE, 
  fornitore numeric(11, 0) REFERENCES fornitore(partitaiva) ON UPDATE CASCADE ON DELETE CASCADE, 
  PRIMARY KEY (dipartimento, fornitore)
);

CREATE TABLE impiegato (
matricola integer PRIMARY KEY, 
nome varchar (20), 
cognome varchar (20), 
data_di_nascita date, 
data_di_assunzione date, 
dipartimento varchar(50) REFERENCES dipartimento(nome) ON UPDATE CASCADE ON DELETE RESTRICT, 
qualifica varchar (20), 
numero_progetti integer NOT null DEFAULT 0
);

CREATE TABLE segretario (
impiegato integer REFERENCES impiegato(matricola) ON UPDATE CASCADE ON DELETE CASCADE, 
lingua lingua, 
PRIMARY KEY (impiegato, lingua)
);

CREATE TABLE competenza (
  codice integer PRIMARY KEY, 
  descrizione varchar (50) UNIQUE
);

CREATE TABLE citta (
  nome varchar (20), 
  regione varchar (20), 
  numero_di_residenti integer, 
  PRIMARY KEY (nome, regione)
);

CREATE TABLE progetto (
numero integer PRIMARY KEY, 
budget integer, 
citta varchar(20), 
regione varchar(20), 
FOREIGN KEY(citta, regione) REFERENCES citta(nome, regione) ON UPDATE CASCADE ON DELETE RESTRICT
);

CREATE TABLE matrimonio (
marito integer REFERENCES impiegato (matricola) ON UPDATE CASCADE ON DELETE CASCADE PRIMARY KEY, 
moglie integer REFERENCES impiegato(matricola) ON UPDATE CASCADE ON DELETE CASCADE UNIQUE NOT null, 
data_di_matrimonio date, 
CONSTRAINT marito_moglie_diversi CHECK(marito <> moglie)
);

CREATE TABLE laureato (
impiegato integer REFERENCES impiegato(matricola) ON UPDATE CASCADE ON DELETE CASCADE, 
tipo_laurea varchar (20), 
materia varchar (50), 
PRIMARY KEY (impiegato, tipo_laurea, materia)
);

CREATE TABLE partecipa (
impiegato integer REFERENCES impiegato(matricola) ON UPDATE CASCADE ON DELETE CASCADE, 
competenza integer REFERENCES competenza(codice) ON UPDATE CASCADE ON DELETE RESTRICT, 
progetto integer REFERENCES progetto(numero) ON UPDATE CASCADE ON DELETE RESTRICT, 
PRIMARY KEY(impiegato, competenza, progetto)
);
\end{minted}

\subsection{Definizione delle viste}
Segue la definizione di una vista che è stata formulata per facilitare alcune interrogazioni e l'implementazione del vincolo aziendale relativo alla partecipazione ai progetti da parte degli impiegati.
\begin{minted}{sql}
CREATE VIEW cittaprogetto(progetto, citta, regione) AS 
SELECT 
  p.numero, 
  ct.nome, 
  ct.regione 
FROM 
  progetto p 
  JOIN citta ct ON p.citta = ct.nome 
  AND p.regione = ct.regione;
\end{minted}

\newpage

\subsection{Definizione dei trigger}
Segue la definizione dei trigger relativi ai vincoli aziendali e alla consistenza della base di dati.

\subsubsection{Vincolo aziendale relativo alla partecipazione ai progetti per città}
La funzione con trigger implementa il primo \hyperlink{page.9}{vincolo aziendale}.
\begin{minted}{sql}
CREATE 
OR REPLACE FUNCTION max_progetto_citta() RETURNS TRIGGER LANGUAGE plpgsql AS $$ BEGIN IF (
  new.competenza <> old.competenza 
  AND (
    new.progetto = old.progetto 
    AND new.impiegato = old.impiegato
  )
) THEN RETURN new;
END IF;
IF (
  NOT EXISTS (
    SELECT 
      * 
    FROM 
      progetto, 
      partecipa, 
      cittaprogetto 
    WHERE 
      new.progetto = progetto.numero 
      and new.impiegato = partecipa.impiegato 
      and partecipa.progetto = cittaprogetto.progetto 
      and progetto.citta = cittaprogetto.citta
  )
) THEN RETURN new;
END IF;
Return old;
END $$;

CREATE TRIGGER max_progetto_citta before 
INSERT 
OR 
UPDATE 
  ON partecipa FOR each ROW execute procedure max_progetto_citta();
\end{minted}

\newpage

\subsubsection{Vincolo aziendale relativo agli impiegati con qualifica di segretario}
Le funzioni seguenti implementano il \hyperlink{page.16}{vincolo aziendale sugli impiegati con qualifica di segretari}, con un ulteriore garanzia relativa alla tabella segretario.

\subsubsection*{Vincolo su impiegato}

Un impiegato segretario deve conoscere almeno una lingua quindi la seguente funzione con trigger impedisce l'inserimento di impiegati con qualifica "Segretario", o l'attribuzione della stessa a impiegati che ne possedevano una diversa, se l'impiegato non è già presente nella tabella \textbf{segretario} dove la lingua fa parte della chiave primaria.
\begin{minted}{sql}
CREATE 
OR REPLACE FUNCTION impiegato_segretario() RETURNS TRIGGER LANGUAGE plpgsql AS $$ BEGIN IF (
  new.qualifica = 'Segretario' 
  AND (
    NOT EXISTS (
      SELECT 
        * 
      FROM 
        segretario 
      WHERE 
        impiegato = new.matricola
    )
  )
) THEN RETURN old;
END IF;
RETURN new;
END $$;

CREATE TRIGGER impiegato_segretario before 
INSERT 
OR 
UPDATE 
  ON impiegato FOR each ROW execute procedure impiegato_segretario();
\end{minted}

\newpage

\subsubsection*{Vincolo su segretario}

La funzione, questa volta con un check integrato nella tabella, implementa un ulteriore vincolo la cui utilità è associata a quella del precedente. Questo vincolo è necessario a garantire che nella tabella segretario siano presenti solo impiegati con tale qualifica.
\begin{minted}{sql}
CREATE 
OR REPLACE FUNCTION vincolo_segretario(m integer) RETURNS boolean LANGUAGE plpgsql AS $$ BEGIN IF (
  m NOT IN (
    SELECT 
        matricola 
    FROM 
        impiegato 
    WHERE 
          qualifica = 'Segretario' 
        OR 
          qualifica IS NULL 
  )
) THEN RETURN false;
END IF;
RETURN true;
END $$;

ALTER TABLE segretario 
    ADD CONSTRAINT vincolo_segretario 
        CHECK (vincolo_segretario(impiegato));
\end{minted}
Come è facile intuire la coesistenza dei due trigger precedenti non permette l'inserimento di impiegati con qualifica di segretario o di segretari.
\newline
\newline
Noto che PostgreSQL non permette il deferimento di vincoli di tipo CHECK (e NOT NULL ma non è questo il caso) e volendo rispettare a pieno le specifiche che prevedevano "Segretario" come possibile valore dell'attributo qualifica, si è optato per l'utilizzo del valore temporaneo null come "jolly" per permettere gli inserimenti senza rinunciare a nessuno dei due vincoli.
\newline
\newline
Il comportamento della base di dati può essere chiarito visionando la \hyperlink{page.36}{transazione relativa} all'inserimento di impiegati segretari.

\newpage

\subsubsection{Vincolo relativo ai matrimoni}
La funzione con trigger implementa un vincolo sulla tabella matrimonio che previene ad esempio che un impiegato risulti contemporaneamente sposato con più impiegati diversi o che sia possibile inserire coppie permutate.
\newline
L'eventualità dell'impiegato sposato con se stesso era già gestita con un constraint a livello di \hyperlink{page.27}{tabella}.
\begin{minted}{sql}
CREATE 
OR REPLACE FUNCTION matrimonio_unico() RETURNS TRIGGER LANGUAGE plpgsql AS $$ BEGIN IF (
  EXISTS (
    SELECT 
      * 
    FROM 
      matrimonio 
    WHERE 
      marito = new.marito 
      OR marito = new.moglie 
      OR moglie = new.marito 
      OR moglie = new.moglie
  )
) THEN RETURN old;
END IF;
RETURN new;
END $$;

CREATE TRIGGER matrimonio_unico before 
INSERT 
OR 
UPDATE 
  ON matrimonio FOR each ROW execute procedure matrimonio_unico();
\end{minted}

\newpage

\subsubsection{Sincronizzazione attributo derivato numero fornitori di dipartimento}
Seguono funzioni e trigger necessari per la gestione del primo attributo derivato.
\begin{minted}{sql}
CREATE 
OR replace FUNCTION numero_fornitori_inc() RETURNS TRIGGER LANGUAGE plpgsql AS $$ BEGIN 
UPDATE 
  dipartimento 
SET 
  numero_fornitori = numero_fornitori + 1 
WHERE 
  nome = new.dipartimento;
RETURN new;
END;
$$;

CREATE TRIGGER numero_fornitori_inc before 
INSERT 
    ON rifornisce FOR each ROW execute procedure numero_fornitori_inc();

CREATE 
OR replace FUNCTION numero_fornitori_dec() RETURNS TRIGGER LANGUAGE plpgsql AS $$ BEGIN 
UPDATE 
  dipartimento 
SET 
  numero_fornitori = numero_fornitori - 1 
WHERE 
  nome = old.dipartimento;
RETURN old;
END;
$$;

CREATE TRIGGER numero_fornitori_dec before 
DELETE 
    ON rifornisce FOR each ROW execute procedure numero_fornitori_dec();

CREATE 
OR replace FUNCTION numero_fornitori_update() RETURNS TRIGGER LANGUAGE plpgsql AS $$ BEGIN 
UPDATE 
  dipartimento 
SET 
  numero_fornitori = numero_fornitori + 1 
WHERE 
  nome = new.dipartimento;
UPDATE 
  dipartimento 
SET 
  numero_fornitori = numero_fornitori - 1 
WHERE 
  nome = old.dipartimento;
RETURN new;
END;
$$;

CREATE TRIGGER numero_fornitori_update before 
UPDATE 
  ON rifornisce FOR each ROW execute procedure numero_fornitori_update();
\end{minted}

\newpage

\subsubsection{Sincronizzazione attributo derivato numero progetti di impiegato}
Seguono funzioni e trigger necessari per la gestione del secondo attributo derivato.
\begin{minted}{sql}
CREATE 
OR replace FUNCTION numero_progetti_inc() RETURNS TRIGGER LANGUAGE plpgsql AS $$ BEGIN 
UPDATE 
  impiegato 
SET 
  numero_progetti = numero_progetti + 1 
WHERE 
  matricola = new.impiegato;
RETURN new;
END;
$$;

CREATE TRIGGER numero_progetti_inc before 
INSERT 
    ON partecipa FOR each ROW execute procedure numero_progetti_inc();

CREATE 
OR replace FUNCTION numero_progetti_dec() RETURNS TRIGGER LANGUAGE plpgsql AS $$ BEGIN 
UPDATE 
  impiegato 
SET 
  numero_progetti = numero_progetti - 1 
WHERE 
  matricola = old.impiegato;
RETURN old;
END;
$$;

CREATE TRIGGER numero_progetti_dec before 
DELETE 
    ON partecipa FOR each ROW execute procedure numero_progetti_dec();

CREATE 
OR replace FUNCTION numero_progetti_update() RETURNS TRIGGER LANGUAGE plpgsql AS $$ BEGIN 
UPDATE 
  impiegato 
SET 
  numero_progetti = numero_progetti + 1 
WHERE 
  matricola = new.impiegato;
UPDATE 
  impiegato 
SET 
  numero_progetti = numero_progetti - 1 
WHERE 
  matricola = old.impiegato;
RETURN new;
END;
$$;
CREATE TRIGGER numero_progetti_update before 
UPDATE 
  ON partecipa FOR each ROW execute procedure numero_progetti_update();
\end{minted}

\newpage

\subsection{Definizione degli indici}
Segue la definizione degli indici analizzati in fase di \hyperlink{page.17}{progettazione fisica}.

\subsubsection{Indice su qualifica}
\begin{minted}{sql}
CREATE INDEX index_impiegato_qualifica ON impiegato(qualifica);

\end{minted}

\subsubsection{Indice su data di assunzione}
\begin{minted}{sql}
CREATE INDEX index_impiegato_dataAssunzione ON impiegato(data_di_assunzione);

\end{minted}

\subsubsection{Indice su budget}
\begin{minted}{sql}
CREATE INDEX index_progetto_budget ON progetto(budget);

\end{minted}

\subsection{Interrogazioni}
Vengono presentate alcune interrogazioni rilevanti costruite sulla base dei \hyperlink{page.17}{requisiti operazionali}.

\subsubsection{Ricerca dei segretari che conoscono una determinata lingua}
\begin{minted}{sql}
SELECT 
  impiegato.matricola, 
  impiegato.nome, 
  impiegato.cognome 
FROM 
  impiegato, 
  segretario 
WHERE 
  impiegato.matricola = segretario.impiegato 
  AND lingua = 'Russo';
\end{minted}

\subsubsection{Ricerca degli impiegati con determinata competenza}
\begin{minted}{sql}
SELECT 
  DISTINCT matricola 
FROM 
  impiegato 
  JOIN partecipa ON impiegato.matricola = partecipa.impiegato 
WHERE 
  partecipa.competenza = 824;
\end{minted}

\subsubsection{Ricerca delle competenze di un impiegato}
\begin{minted}{sql}
SELECT 
  competenza 
FROM 
  partecipa 
WHERE 
  impiegato = 69992;
\end{minted}

\newpage

\subsubsection{Ricerca degli impiegati coniugati}
\begin{minted}{sql}
SELECT 
    matricola 
FROM 
  impiegato 
WHERE 
  matricola IN (
    SELECT 
      marito 
    FROM 
      matrimonio
  ) 
  OR matricola IN (
    SELECT 
      moglie 
    FROM 
      matrimonio
  );
\end{minted}

\subsubsection{Ricerca dei laureati in una materia}
\begin{minted}{sql}
SELECT 
  impiegato 
FROM 
  laureato 
WHERE 
  materia = 'Physics';
\end{minted}

\subsubsection{Ricerca delle città in cui l'azienda opera}
\begin{minted}{sql}
SELECT 
  DISTINCT citta 
FROM 
  progetto;
\end{minted}

\subsubsection{Ricerca del numero di dipendenti in una determinata città}
\begin{minted}{sql}
SELECT 
  COUNT(DISTINCT impiegato) 
FROM 
  partecipa AS p 
  JOIN progetto AS ptt ON p.progetto = ptt.numero 
WHERE 
  citta = 'El Paso';
\end{minted}

\subsubsection{Ricerca del numero di progetti a cui lavora un impiegato}
\begin{minted}{sql}
SELECT 
  numero_progetti 
FROM 
  impiegato 
WHERE 
  matricola = 69992;
\end{minted}

\newpage

\subsubsection{Ricerca del numero di fornitori di un dipartimento}
\begin{minted}{sql}
SELECT 
  numero_fornitori 
FROM 
  dipartimento 
WHERE 
  nome = 'B028';
\end{minted}

\subsubsection{Assegnazione di un progetto ad un impiegato}
\begin{minted}{sql}
INSERT INTO partecipa(impiegato, competenza, progetto) 
VALUES 
  (69992, 3247, 13);
\end{minted}

\subsubsection{Assegnazione di un fornitore a un dipartimento}
\begin{minted}{sql}
INSERT INTO rifornisce(dipartimento, fornitore) 
VALUES 
  ('V8644XS', 23489419167);
\end{minted}

\subsubsection{Inserimento di un impiegato segretario}
A causa dei vincoli, per inserirne un segretario è necessario inserire prima un impiegato con qualifica nulla.
\newline
Dopodiché sarà possibile l'inserimento di un record riferito a quella matricola nella tabella segretario, corredato di una lingua.
\newline
Solo in seguito è possibile aggiornare la qualifica dell'impiegato in segretario.
\newline
\newline
Si prevede quindi l'utilizzo di una transazione:
\begin{minted}{sql}
START TRANSACTION;

INSERT INTO impiegato 
    (matricola, nome, cognome, data_di_nascita, data_di_assunzione, dipartimento, qualifica) 
VALUES 
	(28172, 'Lianna', 'Vitler', '1980/04/25', '2019/05/05', 'C221', null);

INSERT INTO segretario (impiegato, lingua) 
VALUES 
	(28172, 'Russo');

UPDATE 
  impiegato 
SET 
  qualifica = 'Segretario' 
WHERE 
  matricola = 28172;

COMMIT;
\end{minted}
% TODO: essere più specifici, estendere la trattazione di cosa si è provato e perché si è scelto così
Non è possibile automatizzare oltre il processo perché l'informazione relativa alla lingua deve necessariamente essere inserita a mano.

\newpage

\section{Analisi dei dati}
Una volta completata l'implementazione e inseriti i dati di mockup è stato possibile procedere all'analisi dei dati in R. Viene presentato il codice SQL usato per raccogliere i dati seguito da un opportuno grafico prodotto a partire dall'elaborazione degli stessi.
\newline
\newline
Per questo fine sono state utilizzate le librerie RPostgreSQL, dplyr e ggplot2, rispettivamente per la connessione alla base di dati, per la manipolazione dei risultati delle query e per produrre e salvare le visualizzazioni.

\subsection{Numero progetti per città}
A seguito dell'interrogazione sono stati prodotti dei bar plot raffiguranti, per ciascuna città, il numero di progetti attivi.
\begin{minted}{sql}
SELECT citta, COUNT(*) 
FROM cittaprogetto 
GROUP BY citta 
ORDER BY citta;
\end{minted}
\begin{center}
\includegraphics[width=\textwidth]{plot_numero_progetti_citta.png}
\end{center}

\newpage

\subsection{Distribuzione data di nascita per qualifica}
A seguito dell'interrogazione sono stati prodotti dei box plot raffiguranti, per ciascuna qualifica, la distribuzione della data di nascita degli impiegati.
\newline
Come si può notare la mediana delle date di nascita degli analisti, impiegati, programmatori e segretari si attesta tra gli anni 1980 e 1990 mentre la mediana delle date di nascita dei manager si attesta tra gli anni 1970 e 1980 con la presenza di 2 outlier.
\begin{minted}{sql}
SELECT data_di_nascita, qualifica 
FROM impiegato;
\end{minted}
\begin{center}
\includegraphics[width=\textwidth]{plot_dist_dNascita_qualifica.png}
\end{center}

\newpage

\subsection{Numero segretari per lingua}
A seguito dell'interrogazione sono stati prodotti dei bar plot raffiguranti, per ciascuna lingua, il numero di segretari che conosce quella determinata lingua.
\newline
Come si può notare i segretari dell'azienda rappresentata nella nostra banca dati conoscono un ampio spettro di lingue straniere, con una preminenza di conoscitori della lingua francese, svedese e tedesca.
\begin{minted}{sql}
SELECT lingua, count(*) 
FROM segretario 
GROUP BY lingua 
ORDER BY lingua;
\end{minted}
\begin{center}
\includegraphics[width=\textwidth]{plot_numero_segretari_lingua.png}
\end{center}

\newpage

\subsection{Distribuzione budget per città}
A seguito dell'interrogazione sono stati prodotti dei box plot raffiguranti, per ciascuna città, la distribuzione del budget dei progetti lì svolti.
\newline
Come si può notare la mediana dei budget di tutte le città, esclusa Tampa, si attesta in un intervallo compreso tra i 2,5 milioni e tra i 6 milioni.
\begin{minted}{sql}
SELECT budget, citta 
FROM progetto;
\end{minted}
\begin{center}
\includegraphics[width=\textwidth]{plot_dist_budget_progetto_citta.png}
\end{center}

\newpage

\subsection{Distribuzione partecipazioni a progetto con competenze}
A seguito dell'interrogazione è stato prodotto un box plot raffigurante la distribuzione delle partecipazioni per competenza.
\newline
Come si può notare la mediana di partecipazioni in cui viene applicata una competenza è 300 con una varianza compresa tra 290 e 310.
\newline
Non compaiono dei veri e propri outlier ma alcune competenze sono impiegate meno (270) o più spesso (325).
\begin{minted}{sql}
SELECT descrizione, COUNT(*) FROM competenza, partecipa 
WHERE codice = partecipa.competenza 
GROUP BY codice 
ORDER BY descrizione;
\end{minted}
\begin{center}
\includegraphics[width=.5\textwidth]{plot_dist_partecipazioni_comptenza.png}
\end{center}

\newpage

\subsection{Distribuzione impiegati per dipartimento}
A seguito dell'interrogazione è stato prodotto un box plot raffigurante la distribuzione degli impiegati per dipartimento.
\newline
Come si può notare si ha una mediana di circa 150 impiegati per dipartimento con un outlier prossimo ai 230.
\begin{minted}{sql}
SELECT dipartimento, COUNT(*) 
FROM impiegato, partecipa 
WHERE matricola = partecipa.impiegato 
GROUP BY dipartimento 
ORDER BY dipartimento
\end{minted}
\begin{center}
\includegraphics[width=.5\textwidth]{plot_dist_impiegati_dipartimento.png}
\end{center}

\newpage

\subsection{Percentuale impiegati laureati}
A seguito dell'interrogazione è stato prodotto un diagramma a torta raffigurante il numero di impiegati laureati e non.
\newline
Come si può notare dal grafico il 66.7\% degli impiegati non ha conseguito una laurea, mentre il 33\% degli impiegati risulta laureato.
\begin{minted}{sql}
SELECT COUNT(*) 
FROM laureato 
UNION 
SELECT COUNT(*) 
FROM impiegato 
WHERE NOT EXISTS(
    SELECT * 
    FROM laureato 
    WHERE matricola = laureato.impiegato
    );
\end{minted}
\begin{center}
\includegraphics[width=.96\textwidth]{plot_perc_laureati.png}
\end{center}

\newpage

\subsection{Percentuale impiegati per qualifica}
A seguito dell'interrogazione è stato prodotto un diagramma a torta raffigurante, per ciascuna qualifica, la percentuale di dipendenti impiegati con quella determinata qualifica.
\newline
Come si può notare dal grafico quasi il 50\% degli impiegati sono dei programmatori, mentre i manager ed i segretari costituiscono una parte marginale dell'organico dell'azienda.
\begin{minted}{sql}
SELECT qualifica, COUNT(*)/(SUM(COUNT(*)) OVER()) AS frequenza 
FROM impiegato 
GROUP BY qualifica;
\end{minted}
\begin{center}
\includegraphics[width=\textwidth]{plot_perc_impiegati_qualifica.png}
\end{center}

\newpage

\section{Conclusione}
La relazione descrive la progettazione e implementazione di una base di dati relazionale per un'ipotetica azienda.
\newline
Il processo è partito dai requisiti forniti poi formalizzati in specifiche.
\newline
Particolare attenzione è stata rivolta alla fase di progettazione concettuale.
\newline
Per la realizzazione degli schemi è stato usato \href{https://www.diagrams.net}{Diagrams.net}.
\newline
Si è sfruttata la libreria \href{https://www.mockaroo.com/docs}{Mockaroo} per generare dati di mockup il più possibile adatti al dominio e significativi.
\newline
Ulteriori operazioni relative ai dati da inserire sono state svolte con espressioni regolari, bash e Python.
\newline
Per la collaborazione e i test sulla base di dati si sono sfruttati git, \href{https://www.overleaf.com/}{Overleaf}, container e virtualizzazione.
\newline
L'analisi dei dati in R è stata per quanto possibile automatizzata nella generazione di csv e grafici.
\newline
\newline
In tutte le fasi del progetto si è fatto uso delle conoscenze acquisite durante il corso.
\end{document}

